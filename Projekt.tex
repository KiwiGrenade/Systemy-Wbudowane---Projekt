\documentclass{article}
\usepackage[utf8]{inputenc}
\usepackage[T1]{fontenc}

\title{Wykrywacz Metali - Dokumentacja}
\author{
	Jakub Jaśków
	\and
	Justyna Ziemichód
}
\date{\today}

\begin{document}

\maketitle

\section{Funkcje Systemu}
\begin{itemize}
    \item Wykrywanie metali.
    \item Zmiana tonacji dźwięku.
    \item Czujnik odległości: Szacuje głębokość na której znajduje się wykryty obiekt.
    \item Wybór systemu powiadomiania: dźwięk, wibracje, LED
    \item Wybór czułości wykrywacza: większa czułość pozwala na wykrycie mniejszych lub słabiej przewodzących prąd obiektów. Skutkuje to jednak większą szansą odebrania zakłóceń.
    \item Wybór rodzaju metalu (dyskryminacja): Użytkownik może wybrać wykrywany rodzaj metalu (żelazo, złoto, srebro itp.).
\end{itemize}

\section{Realizacja Założeń}
	
- Wykrywanie metali: Cewka wykrywacza wytwarza fale elektromagnetyczne, które przenikają do gruntu, oraz dobera fale elektromagnetyczne wzbudzone w obiekcie, które odbierane są przez tą samą cewkę. Sygnał jest następnie przekazywane do mikrokontrolera, który zamienia go na dane, które są w odpowieni sposób przetwarzane. Jeżeli mikrokontroler wykryje, że fale elektromagnetyczne zostały zakłócone w sposób odpowiadający wybranemu rodzaju metalu, to poinformuje użytkownika w wybranych przez niego sposób.
\section{Środowisko i Ograniczenia}
System może być używany w różnych warunkach, zarówno wewnątrz pomieszczeń, jak i na zewnątrz. Jednakże jego skuteczność może być ograniczona przez warunki otoczenia, takie jak obecność innych źródeł metalu, zakłócenia elektromagnetyczne, mineralizacja gleby lub woda.

\section{Kategorie Użytkowników}
System wykrywacza metali może być wykorzystywany przez różne grupy użytkowników, w tym:
\begin{itemize}
    \item \textbf{Ochroniarze}: Używają wykrywacza do przeszukiwania osób i bagaży w celu wykrycia niepożądanych przedmiotów metalowych.
    \item \textbf{Poszukiwacze Skarbów}: Wykorzystują go do poszukiwania ukrytych obiektów metalowych, takich jak monety czy artefakty.
    \item \textbf{Geodeci}: Mogą używać go do wykrywania ukrytych rur lub kabli pod ziemią.
    \item\textbf{Śledczy}: Wykorzystują go do poszukiwania poszlak lub narzędzi zbrodni.
\end{itemize}

\end{document}
